% ----------------------------------------------------------
% Introdução 
% 
% ----------------------------------------------------------

\chapter[Introdução]{Introdução}

Gerenciar as finanças pessoais é uma tarefa que pode ser difícil e desafiadora, mas com a ajuda da tecnologia, torna-se mais fácil e acessível. O sistema Financial Manager tem como objetivo fornecer uma solução eficaz para ajudar os usuários a controlar suas finanças de forma simples e intuitiva. O FiMa é desenvolvido para as plataformas móveis \textit{Android} e \textit{iOS}, utilizando tecnologias e ferramentas atualizadas e relevantes para a criação de aplicativos móveis com alto grau de qualidade. Este trabalho apresentará o processo de desenvolvimento do FiMa e as pesquisas relacionadas, bem como suas funcionalidades e características.

\section{Objetivos}

Neste TCC temos como objetivo buscar uma solução funcional em forma de aplicativo para todas as pessoas que tem a necessidade de melhorar ou criar um controle das suas finanças. Proporcionando de uma forma fácil e didática o incremento de valores e gastos na plataforma criada. Através de recursos tecnológicos, esta aplicação tem como objetivos:
\begin{enumerate}
    \item Criar uma pesquisa com os possíveis usuários, identificando suas principais necessidades e desejos em um aplicativo, entendendo os principais tipos de renda que procuram um aplicativo de controle financeiro e a idade dos mesmos.
    \item Promover a educação financeira: Além de ajudar os usuários com suas finanças básicas, nosso aplicativo tem como objetivo educar os usuários sobre as finanças pessoais, oferecendo dicas sobre como economizar seu dinheiro e sempre alertando o mesmo se seus gastos estão acima dos seus ganhos.
    \item Facilitar o controle financeiro dos usuários: O aplicativo vem com o principal objetivo de ajudar os usuários a controlarem suas finanças de uma maneira mais consciente, oferecendo ferramentas como o registro das despesas do mês, gastos programados.
    \item Desenvolver um \textit{Back End} funcional e seguro, com autenticação para todos os endpoints, pois estaremos trabalhando com dados pessoais e devemos evitar qualquer vazamento.
    \item Desenvolver um \textit{Front End} intuitivo, que o usuário entenda bem como utilizar todos os pedaços da aplicação.
    \item Definir os recursos funcionais e não funcionais da aplicação.
    \item O aplicativo será testado em todos os pedaços, levando as metodologias Ágeis como foco, sempre em contato com os usuários e avaliando se o que foi desenvolvido está de acordo com o esperado.
    \item Apresentação do projeto para o orientador \imprimirorientador.
    \item Entrega final do aplicativo para a banca da Universidade Municipal de São Caetano do Sul.
\end{enumerate}

\section{Justificativa}

O projeto FiMa tem como objetivo desenvolver um aplicativo de gerenciamento financeiro. Este aplicativo auxiliará os usuários a terem uma organização financeira mais eficiente, representando uma contribuição original para as áreas de tecnologia e finanças.

\subsection{Importância do trabalho para o mercado}
O mercado financeiro está passando por uma transformação significativa com a nova era digital. Nesse sentido, um aplicativo de gerenciamento financeiro que aproveite as vantagens dessa integração terá um papel relevante no cenário atual. O FiMa tem a capacidade de oferecer aos usuários uma solução eficiente e segura para gerenciar suas finanças.

\subsection{Importância do trabalho para a sociedade}
O acesso a informações financeiras de forma transparente e simplificada é de grande importância para os indivíduos e para a sociedade como um todo. Os usuários terão a oportunidade de tomar decisões mais assertivas sobre suas finanças pessoais, planejar melhor seu orçamento, otimizar investimentos e controlar gastos. Isso pode levar a uma maior educação financeira e bem-estar econômico para os usuários, contribuindo para o desenvolvimento socioeconômico da comunidade. Dessa forma, o projeto se justifica por sua relevância para validar um aplicativo de gerenciamento financeiro, considerando sua importância tanto para o mercado quanto para a sociedade.

\section{Delimitação do estudo}

\begin{enumerate}
    \item Este TCC tem como delimitação o desenvolvimento de um aplicativo de controle financeiro, que tem como usuários todos os interessados em melhorar sua organização financeira.
    \item Este software será desenvolvido para ambas as plataformas de sistemas, sendo elas \textit{IOS} e \textit{Android} e publicado nas lojas de aplicativos respectivas.
    \item O aplicativo terá funções básicas de \textit{CRUD} para adicionar novas despesas, podendo classificar esses gastos em diversos temas como casa, lazer, trabalho, viagem e etc. Podendo verificar todos os gastos e ganhos durante o mês, separando no que teve mais movimentação, sendo gastos ou ganhos do mês.
    \item O estudo irá se concentrar em um publico alvo determinado pela pesquisa de campo que será feita para toda a faculdade, assim obtendo os possíveis interessados em nosso software.
    \item Utilizaremos referenciais bibliográficas de pesquisas e artigos de autores que são referência no assunto de finanças ou software, assim como livros com foco no assunto de controle de finanças.
\end{enumerate}

