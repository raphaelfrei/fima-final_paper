% ----------------------------------------------------------
% Considerações Finais
% 
% ----------------------------------------------------------

\chapter[Resultados finais]{Resultados Finais}

O presente trabalho adota uma abordagem de finalização que se baseia em um protótipo, ou seja, uma representação do produto final que tem como finalidade proporcionar uma experiência imersiva do aplicativo, sendo assim, os usuários poderão interagir dentro dele e conhecer na prática sua proposta. O projeto foi organizado por uma série de etapas planejadas, cada uma representando um marco significativo em nossa jornada a uma conclusão bem-sucedida. A primeira etapa foi da pesquisa bibliográfica, possibilitando maior conhecimento sobre os temas da pesquisa. A segunda foi a pesquisa de cunho quantitativo com possíveis usuários, a partir dos dados obtidos através da realização da pesquisa, foi possível planejar nosso projeto de forma mais assertiva para atender a necessidade de todos os usuários e viabilizar uma melhor experiência. Em seguida a criação do logo, paleta de cores, primeiras telas do aplicativo no Figma, início do desenvolvimento dos códigos e primeiras interações com o aplicativo funcionando. 

O plano de teste foi pensado estritamente para analisar os principais casos e possíveis erros, um componente essencial e estratégico para assegurar a qualidade e confiabilidade do projeto em questão. Portanto, a partir dos testes planejados e executados, os erros apresentados inicialmente puderam ser corrigidos, logo concluímos essa fase com êxito.
O desenvolvimento deste trabalho foi uma etapa crucial, onde se tornou evidente a aplicação prática das teorias e métodos propostos inicialmente. Ao longo deste processo, nosso principal objetivo foi criar um aplicativo para ajudar pessoas que estão buscando uma maneira mais fácil e prática para organizar sua vida financeira, além de contribuir com a transformação que tem ocorrido no mercado financeiro. Para a construção do design de interfaces utilizamos o Figma (UX/UI), para a criação da tela (\textit{Front End}) foi utilizado o .NET MAUI e para implementar suas funcionalidades (\textit{Back End}) o C\#. 

Nosso Ambiente de desenvolvimento Integrado (IDE) foi o Visual Studio, Android Studio, que é um emulador Android, no qual simula um celular com o sistema operacional Android e o XCode no qual simula um celular com sistema operacional iOS, da Apple. O principal desafio enfrentado por nossa equipe, foi desenvolvendo as telas do aplicativo, no qual estávamos utilizando o programa Unity para realizar este processo, porém após finalizar desenvolvimentos sempre nos deparávamos com erros e problemas de comunicação entre as telas assim optando por trocar toda a linguagem \textit{Front End} da aplicação durante o TCC. 

Ao finalizar este projeto, é evidente que cada etapa percorrida contribuiu de maneira significativa para o alcance dos objetivos propostos. O resultado obtido como a finalização do protótipo totalmente funcional, não apenas valida todos os objetivos apontados desde o início, mas também revelam oportunidades de aprimoramento e desenvolvimento futuros. Assim como, a integração do \textit{Open Banking} ao aplicativo, para termos atualizações em tempo real da conta bancária de nossos usuários, aulas sobre finanças com professores qualificados para um melhor entendimento sobre como utilizar seu dinheiro, aulas sobre investimento e ser possível acessar e desbloquear o aplicativo através da biometria cadastrada no dispositivo.

A idealização para a continuidade do projeto seria abrir oportunidade para patrocinadores que gostariam de anunciar suas aulas e possivelmente integrarmos com o Google ADS para anúncios personalizados para cada usuário. Um outro planejamento é existir versões pagas que faz com que não apareçam anúncios para os usuários. Temos uma boa equipe com capacidade de desenvolvimento para fazer com que o projeto não fique apenas com uma prototipação e sim com um aplicativo de fato lançado. Para o projeto se tornar comercial faltam alguns tópicos, um exemplo disso é o que foi citado anteriormente, possibilidade de anúncios no aplicativo, planos pagos pelos usuários, implementação de níveis de segurança melhores, identificação com biometria e muito mais.

Além disso, as lições aprendidas ao longo do caminho se revelaram inestimáveis, mas também muito valiosas para a vida, bem como, trabalho em equipe, gerenciamento de tempo, comprometimento e responsabilidade, gestão de conflitos, adaptações e flexibilidade.