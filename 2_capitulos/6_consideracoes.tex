% ----------------------------------------------------------
% Considerações Finais
% 
% ----------------------------------------------------------

\chapter[Resultados finais]{Resultados Finais}

O presente trabalho adota uma abordagem de finalização que se baseia em um protótipo, ou seja, uma representação do produto final que tem como finalidade proporcionar uma experiência imersiva do aplicativo, sendo assim, os usuários poderão interagir dentro dele e conhecer na prática sua proposta. O projeto foi organizado por uma série de etapas planejadas, cada uma representando um marco significativo em nossa jornada a uma conclusão bem-sucedida. A primeira etapa foi da pesquisa bibliográfica, possibilitando maior conhecimento sobre os temas da pesquisa. A segunda foi a pesquisa de cunho quantitativo com possíveis usuários, a partir dos dados obtidos através da realização da pesquisa, foi possível planejar nosso projeto de forma mais assertiva para atender a necessidade de todos os usuários e viabilizar uma melhor experiência. Em seguida a criação do logo, paleta de cores, primeiras telas do aplicativo no Figma, início do desenvolvimento dos códigos e primeiras interações com o aplicativo funcionando. 

O plano de teste foi pensado estritamente para analisar os principais casos e possíveis erros, um componente essencial e estratégico para assegurar a qualidade e confiabilidade do projeto em questão. Portanto, a partir dos testes planejados e executados, os erros apresentados inicialmente puderam ser corrigidos, logo concluímos essa fase com êxito.
O desenvolvimento deste trabalho foi uma etapa crucial, onde se tornou evidente a aplicação prática das teorias e métodos propostos inicialmente. Ao longo deste processo, nosso principal objetivo foi criar um aplicativo para ajudar pessoas que estão buscando uma maneira mais fácil e prática para organizar sua vida financeira, além de contribuir com a transformação que tem ocorrido no mercado financeiro. 

Para o desenvolvimento da aplicação, a ferramenta Figma foi empregada para prototipagem, enquanto a linguagem C\# foi selecionada para a programação. 

O Figma, uma plataforma colaborativa para a construção de designs e protótipos de interfaces, foi adotado visando à criação de protótipos da aplicação e à consideração das premissas de experiência do usuário.

A escolha do C\# para o desenvolvimento da aplicação foi motivada pela sua robustez e potencial tanto para o \textit{Front End} quanto para o \textit{Back End}. C\# faz parte da plataforma .NET da Microsoft, que oferece uma extensa biblioteca de recursos facilitando o desenvolvimento.

No \textit{Front End}, optou-se por utilizar Xamarin, uma parte da biblioteca .NET, permitindo a criação simultânea de aplicativos para diversas plataformas com um único desenvolvimento. Isso possibilitou uma resposta consistente em todas as plataformas, garantindo uma experiência uniforme.

No \textit{Back End}, o C\# foi utilizado com suas principais bibliotecas e assets para criar um código eficiente, permitindo acessos seguros e eficientes a bancos de dados. Essas escolhas tecnológicas foram fundamentais para o desenvolvimento eficaz da aplicação, garantindo qualidade e desempenho em todas as suas funcionalidades.

O Ambiente de Desenvolvimento Integrado (IDE) utilizado incluiu o Visual Studio, Android Studio (como emulador Android), que simula um dispositivo com o sistema operacional Android, e o XCode, que simula um dispositivo com o sistema operacional iOS da Apple. O principal desafio enfrentado pela equipe envolveu o desenvolvimento das telas do aplicativo, utilizando o programa Unity. Após a conclusão do desenvolvimento, surgiram frequentemente erros e problemas de comunicação entre as telas, levando à decisão de substituir toda a linguagem \textit{Front End} da aplicação durante o Trabalho de Conclusão de Curso (TCC).

Ao finalizar este projeto, fica evidente que cada etapa percorrida contribuiu de maneira significativa para o alcance dos objetivos propostos. O resultado obtido, representado pela finalização do protótipo totalmente funcional, não apenas valida todos os objetivos estabelecidos desde o início, mas também revela oportunidades de aprimoramento e desenvolvimento futuros. A integração do \textit{Open Banking} ao aplicativo permite atualizações em tempo real da conta bancária dos usuários, a oferta de aulas sobre finanças com professores qualificados para proporcionar um melhor entendimento sobre o uso do dinheiro, aulas sobre investimento, e a capacidade de acessar e desbloquear o aplicativo por meio da biometria cadastrada no dispositivo.

A concepção para a continuidade do projeto envolve a abertura de oportunidades para patrocinadores interessados em anunciar suas aulas, possivelmente integrando-se com o Google ADS para anúncios personalizados para cada usuário. Outro plano inclui a existência de versões pagas que eliminam anúncios para os usuários. A equipe possui habilidades de desenvolvimento capazes de transformar o projeto não apenas em uma prototipação, mas em um aplicativo efetivamente lançado. Para a comercialização do projeto, alguns tópicos ainda necessitam ser abordados, como a possibilidade de anúncios no aplicativo, planos pagos pelos usuários, implementação de níveis de segurança aprimorados, identificação com biometria, entre outros.

Adicionalmente, as lições aprendidas ao longo do caminho se revelaram inestimáveis, contribuindo não apenas para o âmbito profissional, mas também para aspectos como trabalho em equipe, gerenciamento de tempo, comprometimento e responsabilidade, gestão de conflitos, adaptações e flexibilidade.