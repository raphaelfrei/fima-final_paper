% ----------------------------------------------------------
% Revisão da Literatura
% 
% ----------------------------------------------------------

\chapter[Referencial Teórico]{Referencial Teórico}

Nesta seção, o objetivo é apresentar os conceitos fundamentais que embasam a realização deste trabalho.

\section{Engenharia de Software}
\nocite{sommerville2007sommerville}
De acordo com Sommerville (2007), a Engenharia de Software é uma disciplina que abrange princípios, ferramentas, métodos e técnicas para desenvolver sistemas de software de alta qualidade.

A Engenharia de Software está presente em todo o ciclo de vida do software, ou seja, desde a concepção, levantamento dos requisitos, até o projeto, testes, implementação e manutenção do sistema. De acordo com Sommerville (2007), a Engenharia de Software é uma disciplina que abrange princípios, ferramentas, métodos e técnicas para desenvolver sistemas de software de alta qualidade.

O ciclo de vida do projeto é um conceito amplo que visa compilar uma sequência de tarefas, atividades e fases realizadas de forma sistemática e uniforme, com quatro objetivos: 

\begin{enumerate}
    \item  Estabelecimento das tarefas a serem realizadas e sua sequência;
    \item Determinar os modelos descritivos do sistema que deverão ser desenvolvidos;
    \item Coordenar as atividades dos participantes como um todo;
    \item Definir critérios para supervisionar e avaliar os resultados e tarefas do projeto.
\end{enumerate}

\nocite{elmasri2000fundamentals}
Elmasri \textit{et al.} (2000) destaca a importância da modelagem de dados ao descrever a estrutura e as relações dos dados que serão utilizados pelo sistema, permitindo um entendimento claro de cada dado e como eles serão armazenados, manipulados e acessados.

No contexto da Engenharia de Software, a modelagem de dados desempenha um papel crucial no desenvolvimento de sistemas de software. Ela se concentra em descrever a estrutura e as relações dos dados que serão utilizados pelo sistema. Através da modelagem de dados, é possível compreender a natureza e o significado de cada dado, bem como definir como eles serão armazenados, manipulados e acessados.
\nocite{teorey2011database}
Conforme Teorey \textit{et al.} (2011), o dicionário de dados desempenha um papel essencial no processo de desenvolvimento de sistemas, fornecendo uma descrição detalhada de cada elemento de dados utilizado no sistema, incluindo definição, tipo de dado, restrições aplicáveis e relacionamentos.

\nocite{silberschatz2011database}
Silberschatz, Korth, Sudarshan (2011) definem que a modelagem de dados envolve técnicas como a modelagem entidade-relacionamento, que permite representar as entidades, atributos e relacionamentos do sistema, bem como a modelagem relacional, que utiliza conceitos tabulares para representar os dados em um formato relacional.
Outra abordagem popular é a modelagem relacional, que utiliza conceitos como tabelas, chaves primárias, chaves estrangeiras e relacionamentos para representar os dados em um formato tabular. Essa abordagem é amplamente utilizada em bancos de dados relacionais e fornece uma base sólida para o armazenamento e manipulação dos dados.
\nocite{garcia2008database}
Garcia-Molina (2008) menciona a importância da definição de restrições e regras de integridade na modelagem de dados, visando garantir a consistência e a qualidade dos dados, evitando inconsistências, erros e problemas de integridade.

\section{Gestão Financeira Pessoal}
\nocite{baptista2015metodologias}
\nocite{cherobim2010financcas}
A gestão das finanças pessoais é de grande importância na vida das pessoas. Envolve a aplicação de princípios financeiros em contextos familiares ou individuais, como definido por Cherobim e Espejo (2010). Trata-se de uma abordagem que leva em conta as circunstâncias específicas de cada pessoa, possibilitando o planejamento e a tomada de decisões adequadas. Dentro da área das finanças pessoais, é crucial considerar os aspectos particulares de cada situação, o que implica a necessidade de avaliar individualmente as circunstâncias para então elaborar estratégias e escolhas adequadas. Uma das práticas associadas a esse campo é o controle das despesas domésticas, que tem como objetivo melhorar a qualidade de vida tanto no plano familiar quanto individual.

\subsection{Conceitos e Teorias de Gestão Financeira Pessoal}

A Gestão financeira pessoal atinge a administração das finanças de uma pessoa ou até mesmo da família, com objetivo de alcançar metas financeiras de curto e longo prazo.
Segundo Cherobim e Espejo (2010), algumas ferramentes de Gestão Financeira são:

\begin{enumerate}
    \item Orçamento que é uma ferramenta crucial na gesta financeira pessoal;
    \item Gestão de fluxo de caixa onde ajuda a compreender a origem e destino do dinheiro;
    \item Planejamento financeiro estabelece metas financeiras a curto e longo prazo, onde você consiga elaborar estratégias para alcançá-las;
\end{enumerate}

\subsection{Importância da Gestão Financeira Pessoal na Vida Individual}

\nocite{ferreira2017importancia}
De acordo com Ferreira (2016), o Brasil registrou cerca de 58,3 milhões de indivíduos em situação de inadimplência. Esse número representa um aumento de 700 mil pessoas em comparação com janeiro do mesmo ano. Segundo informações do ENEF, o que tem ocorrido é que nos últimos tempos, as pessoas têm experimentado um certo desenvolvimento econômico.

Segundo Cherobim e Espejo (2010), a cada ano, torna-se mais acessível para muitos realizar uma variedade de transações financeiras. Os bancos estão ampliando o leque de opções de crédito, o que, por sua vez, torna mais complexo o entendimento das condições para essa facilidade. Embora essa melhora econômica seja positiva, as pessoas estão adquirindo acesso a situações financeiras que anteriormente não tinham. Sem um conhecimento básico, essa acessibilidade pode ter consequências desastrosas para a vida individual, da família e até mesmo para a saúde financeira do país.

\subsection{Abordagens para alcançar o Bem-Estar Financeiro}

O bem-estar financeiro significa a sensação de estar satisfeito em relação a situação financeira da pessoa. Verificamos diversos fatores comportamentais, como excesso de compras, materialismo, comportamento financeiro e conhecimento financeiro. Verificamos que para alcançar o Bem-Estar Financeiro, precisamos tomar cuidado com excesso de compras, investimentos a serem usados, estudar sobre finanças e ter controle com os gastos. Ter controle sobre os gastos, significa você saber lidar com o que está gastando e não deixar ser controlado pelo dinheiro. Com que consiga realizar gastos que futuramente aproveitem melhor a vida (CHEROBIM e ESPEJO, 2010).

\subsection{Aplicativos de Gestão Financeira Pessoal: Integração da Tecnologia}
\nocite{leitao}
Segundo Leitão (2021), no contexto de administração das finanças pessoais, os sistemas digitais desempenham um papel crucial na simplificação e otimização de processos. No qual a tecnologia combinada com a gestão financeira trouxe diversas vantagens. Nos dias de hoje existem inúmeros aplicativos de finanças pessoais, tais como: Mobills e Orçamento Fácil. Estes citados anteriormente tem um pequeno impasse, são exclusivos para uma única plataforma, Mobills é exclusivo da plataforma iOS e Orçamento fácil do Android. A FiMa tem estudado a possibilidade de lançar o aplicativo disponível em todas as plataformas, para assim, alcançar uma maior quantidade de usuários.