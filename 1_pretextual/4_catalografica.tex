% ---
% Ficha Catalográfica
% ---
% Isto é um exemplo de Ficha Catalográfica, ou ``Dados internacionais de
% catalogação-na-publicação''. Você pode utilizar este modelo como referência. 
% Porém, talvez a biblioteca lhe forneça um PDF
% com a ficha catalográfica definitiva após a defesa do trabalho. Quando estiver
% com o documento, salve-o como PDF no diretório do seu projeto e substitua todo
% o conteúdo de implementação deste arquivo pelo comando abaixo:
%
%\begin{fichacatalografica}
 %\center
    %\vspace*{\fill}	
    %{\includegraphics[scale=1]{figs/ficha_catalografica.pdf}}
 %\end{fichacatalografica}
 
% ---

\begin{fichacatalografica}
	\sffamily
	\vspace*{\fill}					% Posição vertical
	\begin{center}					% Minipage Centralizado
	\fbox{\begin{minipage}[c][9cm]{16.4cm}		% Largura
	\small
	
	\hspace{0.5cm} \imprimirtitulo { /} \imprimirautor. --
	\imprimirlocal, \imprimirdata.
	
	\hspace{0.5cm} \thelastpage f. : il. ; 30 cm.\\
	
	\hspace{0.5cm} \imprimirorientadorRotulo~\imprimirorientador\\
	
	\hspace{0.5cm}
	\parbox[t]{\textwidth}{\imprimirtipotrabalho~--~\imprimirinstituicao,
	\imprimirdata.}\\
	
	\hspace{0.5cm}
		1. Financial Manager. % Digitar as palavras-chave em ordem
		2. FiMa.
            3. Gerenciamento financeiro.
            4. Educação financeira.
            5. Gestão financeira pessoal
		I. \imprimirorientador.
		II. Universidade Municipal de São Caetano do Sul.
		III. Ciência da Computação. % Especificar Curso
		IV. \imprimirtitulo{}\\ 			
	\end{minipage}}
	\end{center}
\end{fichacatalografica}